\addchap{Введение}

В повседневной жизни люди нередко сталкиваются с ситуациями, в которых помощь окружающих может значительно облегчить решение какой-либо проблемы. При этом речь может идти как об обычном наличии предметов или продуктов, так и о профессиональной помощи в экстренной ситуации. Для этого нужно обмениваться информацией о своих потребностях или возможностях с другими людьми, самые территориально близкие из которых -- соседи. 

Можно пытаться писать каждому из них со своей просьбой, но гораздо проще разместить ее в специально созданном для этого сервисе. Так как в нынешнее время такой мессенджер как telegram \cite{telegram} стремительно набирает популярность среди населения, целесообразным будет ориентироваться именно на эту площадку.

Целью курсовой работы является проектирование и разработка базы данных для сбора и обработки заявок пользователей, интерфейсом для которой станет telegram-бот.

Для достижения поставленной цели, необходимо решить следующие задачи:
\begin{itemize}
	\item формализовать задачу и определить необходимый функционал;
	\item описать структуру объектов БД, а также сделать выбор СУБД для ее хранения и взаимодействия;
	\item создать БД и заполнить ее необходимым объемом данных;
	\item спроектировать и реализовать интерфейс доступа в формате telegram-бота;
	\item провести тестирование разработанного продукта.
\end{itemize}

Итогом работы станет база данных с разработанным интерфейсом взаимодействия с использованием различных ролевых моделей. Также должно быть предусмотрено наличие триггера на добавление новых заявок в систему.