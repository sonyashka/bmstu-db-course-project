\chapter{Экспериментальный раздел}

В данном разделе будет произведена постановка эксперимента, представлены результаты его проведения, а также сделаны выводы на основе полученных данных.

\section{Цель эксперимента}

Целью эксперимента является оценка изменения времени выполнения операции INSERT в таблицу Request при отсутствии и наличии триггера, используемого для проверки параметров создаваемой заявки.

\section{Описание эксперимента}

Замеры времени добавления заявки в систему осуществлялись при отсутствии триггера и при его наличии, причем при размерностях таблицы Ban 0, 10, 20, 50 строк.

Так как проверка осуществляется по двум полям таблицы Request, имеет место несколько ситуаций: бан-слово в поле заголовка, бан-слово в поле пояснения или отсутствие бан-слова. Первые две ситуации выделяются отдельно, так как от этого зависит количество выполняемых проверок в условии функции триггера, код которой представлен в листинге \ref{trigger_code}.

В таблице \ref{time_measurements} представлены временные замеры по вышеупомянутым параметрам. Каждое значение получено усреднением результатов по 10 замерам. По данной таблице построен график (рисунок \ref{time_graphic}) отображающий исследуемые зависимости.

\begin{table}[H]
	\centering
	\caption{Временные замеры эксперимента (в мс)}
	\label{time_measurements}
	\begin{tabular}{|p{2.6cm}|c|p{1.6cm}|p{1.6cm}|p{1.6cm}|p{1.6cm}|}
		\hline
		\multirow{2}{*}{\textbf{Ситуация}} & \multirow{2}{*}{\textbf{Без триггера}} & \multicolumn{4}{c|}{\textbf{С триггером. Кол-во бан-слов:}}\\
		\cline{3-6}
		& & \textbf{0} & \textbf{10} & \textbf{20} & \textbf{50}\\
		\hline
		Бан-слово в заголовке & 0.1425 & 0.2073 & 0.4291 & 0.4515 & 0.4740 \\
		\hline
		Бан-слово в пояснении & 0.1555 & 0.2008 & 0.4434 & 0.4632 & 0.4901 \\
		\hline
		Отсутствие бан-слова & 0.1491 & 0.1992 & 0.2566 & 0.2705 & 0.2829 \\
		\hline
	\end{tabular}
\end{table}

\begin{figure}[H]
	\captionsetup{singlelinecheck = false, justification=centering}
	\centering
	\begin{tikzpicture}
	\begin{axis}[
	xlabel={количество слов в таблице Ban},
	ylabel={время, мc},
	width=0.95\textwidth,
	height=0.35\textheight,
	xmin=-2, xmax=51,
	ymin=0.1, ymax=0.7,
	legend pos=north west,
	xmajorgrids=true,
	ymajorgrids=true,
	grid style=dashed,
	]
	\addplot[
	color=magenta,
	mark=asterisk
	]
	table [x=N, y=time]{
		N time
		-1 0.1491
		0 0.1992
		10 0.2566
		20 0.2705
		50 0.2829
	};
	\addplot[
	color=blue,
	mark=asterisk
	]
	table [x=N, y=time]{
		N time
		-1 0.1425
		0 0.2073
		10 0.4291
		20 0.4515
		50 0.4740
	};
	\addplot[
	color=red,
	mark=asterisk
	]
	table [x=N, y=time]{
		N time
		-1 0.1555
		0 0.2008
		10 0.4434
		20 0.4632
		50 0.4901
	};
	\legend{Отсутствие бан-слова, Бан-слово в заголовке, Бан-слово в пояснении}
	\end{axis}
	\end{tikzpicture}
	\caption{Временная зависимость добавления заявки в систему от наличия триггера и размерности таблицы Ban}
	\label{time_graphic}
\end{figure}

Из результатов эксперимента можно сделать вывод, что при отсутствии бан-слов время на добавление новой заявки в систему увеличивается до 182\% в зависимости от размеров таблицы Ban. Данный прирост обусловлен получением значений всех бан-слов.

При наличии искомых слов рост времени операции вставки значений в таблицу Request увеличивается на 204-215\%. Такой скачок обусловлен наличием операции обновления поля status проверяемой заявки. Также стоит отметить, что последовательность проверки полей на наличие слова приводит к разбросу в 9\%, которым можно пренебречь относительно общего прироста времени выполнения. 

\section{Вывод из раздела}

В данном разделе был проведен эксперимент относительно времени добавления заявки в систему при отсутствии и наличии триггера на проверку параметров заявки на нецензурную лексику. В ходе эксперимента было получено, что при отсутствии бан-слов время добавления заявки увеличивается в среднем в 1.5 раз, а при наличии -- в среднем в 3 раза. Однако данный механизм является необходимым для системы общего пользования для предотвращения бескультурья. 