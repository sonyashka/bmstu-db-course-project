\addchap{Заключение}

В ходе выполнения данной работы были выполнены следующие задачи:
\begin{itemize}
	\item формализована задача и определен необходимый функционал;
	\item описана структура объектов БД, а также сделан выбор СУБД для ее хранения и взаимодействия;
	\item создана БД и заполнена необходимым объемом данных;
	\item спроектирован и реализован интерфейс доступа в формате telegram-бота;
	\item проведено тестирование разработанного продукта.
\end{itemize}

Была достигнута поставленная цель: спроектирована и разработана база данных для сбора и обработки заявок пользователей, интерфейсом для которой стал telegram-бот.

Необходимый функционал был реализован. Так как приложение спроектировано в соответствии с принципами <<чистой архитектуры>>, его возможности можно расширять посредством добавления новых сущностей в систему.

Также в ходе работы было выполнено исследование влияния наличия триггера проверяющего бан-слова в полях заявки на время добавления заявок. Результаты показали, что использование триггера увеличивает время операции вставки в таблицу Request в 1.5 раза при отсутствии искомых слов и в 3 раза при их наличии.